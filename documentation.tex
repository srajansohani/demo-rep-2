\documentclass{article}
\usepackage{blindtext}
\usepackage{fancyhdr}
\usepackage[a4paper, total ={6in ,8in}]{geometry}
\usepackage{listings}
\lstdefinestyle{chstyle}
%\usepackage[a4paper, left=1in , right = 1in ,bottom=1in,top=1in]{geometry}
\usepackage{graphicx}
\begin{document}
\pagestyle{fancy}
\fancyhead[L]{{\large\bf{0801CS211089}}}
\fancyhead[R]{{\large\bf{Srajan Sohani}}}
\begin{center}
\begin{Huge}
Srajan Sohani \\ \\
0801CS211081 \\ \\
PP-final project \\ \\
Super Market Billing System\\ \\
\end{Huge}
\end{center}
\section{Aim}
The Aim of this mini project program is to make the work of administrator of the shop and buyer easier.It provides an interface for buyer to buy any product and provides bill and administrator can add products when needed.
\section{Working}
There are 2 types of users the buyer and administrator. The user can select whether he is an administrator or  a buyer from the menu. The  administrator have to login with his credentials. On uploading correct credentials 
\section{Following are the functions used in this project}
\subsection{Menu}
This function makes a menu appear in front of the user and asks user to select his identity buyer or administrator.
\subsection{Administrator}
This function is called if the user selects administrator on menu.It opens Administrator menu where user can select whether to Add  or modify or delete an item.
\subsection{Buyer}
If the user select the buyer option at the menu it gets to the buyer's menu. This function display's that buyer's menu to the user and gives 2 options either to Proceed to purchase or go back to menu.
\subsection{add product}
This function is called if user select add product in administrator menu.This function takes input from user of product code,name,price and discount.Then we insert this data into a text file database.txt.
\subsection{modify product}
This function is called when user selects the option modify the product in administrator's menu.This function takes the product code of the item from the user which needs to be modified. then it takes the new values for the product you want to modify.
\subsection{delete product}
The function is called when the user select the delete a product option from administrator menu. It removes the product whose product key is entered by the user.This removal is done in database.txt.
\subsection{Validate id password}
This function is used to validate id and password of the administrator.
\subsection{Receipt}
 It displays the complete shopping menu where you have to enter product code and quantity to buy. The function creates a final receipt  
\subsection{list}
It shows the list of the product so that the user can choose from the list what to buy and discount and prices.

\section{Code In C++ language:}
\begin{verbatim}[style=chstyle,language=C++]
#include <bits/stdc++.h>
#include <fstream>

using namespace std;

class shopping
{
public:
    // All the attributes of a product are declared here.
    int product_code;
    float price;
    float discount;
    string product_name;
    // All the functaionalities are declared here whcih are defined outside using scope resolution
    void menu();
    void administrator();
    void add_product();
    void buyer();
    void modify_product();
    void list();
    void reciept();
    void delete_product();
    bool validate_id_password();
};

bool shopping ::validate_id_password() // It takes email id and password form user and validates it.
{
    string email;
    string password;
    cout << "\t\t\t Please login \n";
    cout << "\t\t\t Enter Email : ";
    cin >> email;
    cout << "\t\t\t Password : ";
    cin >> password;
    if (email == "srajansohani999@gmail.com" && password == "srajan")
    {
        return 1;
    }
    cout << "\n\t\t INVALID ID PASSWORD\n";
    return 0;
}
void shopping ::menu() // It opens the menu for user
{
m:
    int choice;
    cout << "\t\t\t\t\t____________________________________________________\n";
    cout << "\t\t\t\t\t                                                    \n";
    cout << "\t\t\t\t\t          SuperMarket Main Menu                     \n";
    cout << "\t\t\t\t\t                                                    \n";
    cout << "\t\t\t\t\t____________________________________________________\n";
    cout << "\t\t\t\t\t                                                    \n";
    cout << "\t\t\t\t\t    1.)  Administrator     |\n";
    cout << "\t\t\t\t\t                           |\n";
    cout << "\t\t\t\t\t    2.)  Buyer             |\n";
    cout << "\t\t\t\t\t                           |\n";
    cout << "\t\t\t\t\t    3.)  Exit              |\n";
    cout << "\t\t\t\t\t                           |\n";
    cout << "\n\t\t\t Please Select : ";
    cin >> choice;

    switch (choice)
    {
    case 1:
        // If user is administartor he has to enter login id and password
        if (validate_id_password())
        {
            administrator();
        }
        break;

    case 2:
        buyer();
        break;

    case 3:
        exit(0);
        break;

    default:
        cout << "Please select from given options"; // if the user does not select from given options
    }
    goto m;
}

void shopping::administrator() // opens the administrator menu and takes choice of administartor
{
m:
    cout << "\n\n\n\t\t____________________________________________________|";
    cout << "\n\t\t\t\t                                                    |";
    cout << "\n\t\t\t\t          Administrator Menu                        |";
    cout << "\n\t\t\t\t                                                    |";
    cout << "\n\t\t\t\t____________________________________________________|";
    cout << "\n\t\t\t\t                                                    |";
    cout << "\n\t\t\t\t|______1.) Add the product________|";
    cout << "\n\t\t\t\t|                                 |";
    cout << "\n\t\t\t\t|______2.) Modify the product_____|";
    cout << "\n\t\t\t\t|                                 |";
    cout << "\n\t\t\t\t|______3.) Delete the product_____|";
    cout << "\n\t\t\t\t|                                 |";
    cout << "\n\t\t\t\t|______4.) Back to main menu______|";
    cout << "\n\n\t  Please enter your choice : ";
    int choice = 0;
    cin >> choice;
    // depending on choice from above different methods are called
    switch (choice)
    {
    case 1:
        add_product();
        break;
    case 2:
        modify_product();
        break;
    case 3:
        delete_product();
        break;
    case 4:
        menu();
        break;
    default:
        cout << "Invalid choice";
    }
    goto m; // if invalid choice the we again want same function to repeated
}

void shopping::buyer() // opens buyer menu
{
m:
    cout << "\n\t\t\t\t___________________________________________________|";
    cout << "\n\t\t\t\t                                                   |";
    cout << "\n\t\t\t\t                     Buyer  MEnu                   |";
    cout << "\n\t\t\t\t                                                   |";
    cout << "\n\t\t\t\t___________________________________________________|";
    cout << "\n\t\t\t\t                                                   |";
    cout << "\n\t\t\t\t|______1.) Buy Product_________|";
    cout << "\n\t\t\t\t|                              |";
    cout << "\n\t\t\t\t|______2.) Go Back ____________|";
    cout << "\n\t\t\t\t                           ";
    cout << "\n\t\t  Please enter your choice : ";
    int choice = 0;
    cin >> choice;
    switch (choice)
    {
    case 1:
        reciept();
        break;
    case 2:
        menu();
        break;

    default:
        cout << "Invalid choice";
    }
    goto m;
}

void shopping::add_product()
{
m:
    fstream data; // data object for file handling
    // the below variables are decalred to contain the values aleready present in database.txt file
    // this is done to check that the product code entered by user is aleready present or not.
    int code_to_check;
    int token = 0; // token acts as a flag if the value is present or not.
    float price_to_check;
    float discount_to_check;
    string name_to_check;
    cout << "\n\n\t\t\t Add new product";
    cout << "\n\n\t Product code of product: ";
    cin >> product_code;
    cout << "\n\n\t Product Name of product: ";
    cin >> product_name;
    cout << "\n\n\t Price of the Product: ";
    cin >> price;
    cout << "\n\n\t Discount on product: ";
    cin >> discount;

    data.open("database.txt", ios::in);

    if (!data)
    {
        data.open("database.txt", ios::app | ios::out);
        data << " " << product_code << " " << product_name << " " << price << " " << discount << "\n";
        data.close();
    }
    else
    {
        data >> code_to_check >> name_to_check >> price_to_check >> discount_to_check;
        while (!data.eof())
        {
            if (code_to_check == product_code)
            {
                token++;
            }
            data >> code_to_check >> name_to_check >> price_to_check >> discount_to_check;
        }
        data.close();

        if (token == 1)
        {
            goto m;
        }
        else
        {
            data.open("database.txt", ios::app | ios::out);
            data << " " << product_code << " " << product_name << " " << price << " " << discount << "\n";
            data.close();
        }
    }
    cout << "\t\tRECORD INSERTED"
         << "\n";
}

void shopping::modify_product()
{
    fstream data, data1; // using 2 data objects
    // the reason is that the below code is copying all the data from database to database 1 except the product to modify
    int product_key; // product code of product which is to modify.
    int token = 0;
    // th new attributes for the product to modify
    int new_code;
    float new_price;
    float new_discount;
    string new_name;
    cout << "\n\t\t\t  Modify the record";
    cout << "\n\n\n\t Product code ";
    cin >> product_key;
    data.open("database.txt", ios::in);
    if (!data)
    {
        cout << "\n\n File does not exist.\n";
    }
    else
    {
        data1.open("database.txt", ios::app | ios::out);
        data >> product_code >> product_name >> price >> discount;
        while (!data1.eof())
        {
            if (product_key == product_code)
            {
                cout << "\n\t\t Product new code :";
                cin >> new_code;
                cout << "\n\t\t Name of the Product :";
                cin >> new_name;
                cout << "\n\t\t Price :";
                cin >> new_price;
                cout << "\n\t\t Discount :";
                cin >> new_discount;
                data1 << " " << new_code << " " << new_name << " " << new_price << " " << new_discount << "\n";
                cout << "\n\t\t RECORD EDITED\n";
                token++;
            }
            else
            {
                data1 << " " << product_code << " " << product_name << " " << price << " " << discount << "\n";
            }
            data >> product_code >> product_name >> price >> discount;
        }
        data.close();
        data1.close();
        remove("database.txt");
        rename("dtabase1.txt", "databse.txt");
        if (token == 0)
        {
            "Record Not Found";
        }
    }
}

void shopping::delete_product()
{
    fstream data, data1; // same logic as modify
    int product_key;     // the product code of product to delete.
    int token = 0;
    cout << "\n\n\t   Delete Product";
    cout << "\n\n\t   Product Code";
    cin >> product_key;
    data.open("databse.txt", ios::in);
    if (!data)
    {
        cout << "File doesn't exist";
    }
    else
    {
        data1.open("database1.txt", ios::app | ios::out);
        data >> product_code >> product_name >> price >> discount;
        while (!data.eof())
        {
            if (product_code == product_key)
            {
                cout << "\n\n\t Product deleted Succesfully";
                token++;
            }
            else
            {
                data1 << " " << product_code << " " << product_name << " " << price << " " << discount << "\n";
            }
            data >> product_code >> product_name >> price >> discount;
        }
        data.close();
        data1.close();
        remove("database.txt");
        rename("database1.txt", "database.txt");
        if (token == 0)
        {
            cout << "\n\n Record not found";
        }
    }
}

void shopping ::list() // It displays the lsit of availaible products
{
    fstream data;
    data.open("database.txt", ios::in);
    cout << "\t\t PRODUCT LIST\n";
    cout << "\n\n_______________________________________________\n";
    cout << "Product code\t\tName\t\tPrice\n";
    cout << "\n\n________________________________________________ \n";
    data >> product_code >> product_name >> price >> discount;
    while (!data.eof())
    {
        cout << product_code << "\t\t" << product_name << "\t\t" << price << "\t\t" << discount << "\n";
        data >> product_code >> product_name >> price >> discount;
    }
    data.close();
}

void shopping ::reciept() // It displays the complete shopping menu where you have to enter product code and quantity to buy
{
    fstream data;

    int arrC[100];
    int arrQ[100];
    char choice;
    int current_index = 0;
    float amount = 0;
    float final_price = 0;
    float total = 0;
    data.open("database.txt", ios::in);
    if (!data)
    {
        cout << "\n\n Nothing in Store ";
    }
    else
    {
        data.close();
        list();
        cout << "\n_______________________________\n";
        cout << "\n                               \n";
        cout << "\n     Please Place The Order    \n";
        cout << "                                   ";
        cout << "\n_______________________________\n";
        do
        {
        m:

            cout << "\n\n Enter the Proudct code: ";
            cin >> arrC[current_index];
            cout << "\n\n Enter the product quantity: ";
            cin >> arrQ[current_index];
            for (int i = 0; i < current_index; i++)
            {
                if (arrC[i] == arrC[current_index])
                {
                    cout << "\n\n Duplicate product code. Please try again." << endl;
                    goto m;
                }
            }
            current_index++;
            cout << "\n\n Do you want to buy another product ? if yes y else no n : ";
            cin >> choice;
        } while (choice == 'y');

        cout << "\n\n\t\t\t_______________RECIEPT_____________\n";
        cout << "\nProduct No\t Product Name\t Product Quantity\tprice\tAmount\t discount\t Final price\n";
        for (int i = 0; i < current_index; i++)
        {
            data.open("database.txt", ios::in);
            data >> product_code >> product_name >> price >> discount;
            while (!data.eof())
            {
                if (product_code == arrC[i])
                {
                    amount = price * arrQ[i];
                    final_price = amount - (amount * discount / 100);
                    total = total + final_price;
                    cout << "\n"
                         << product_code << "\t\t " << product_name << "\t\t\t" << arrQ[i] << "\t\t" << price << "\t" << amount << "\t\t" << discount << "%\t\t" << final_price;
                }
                data >> product_code >> product_name >> price >> discount;
            }
            data.close();
        }
    }
    cout << "\n\n______________________________________";
    cout << "\n Total Amount : " << total;
}

int main()
{
    shopping s;
    s.menu();
}
\end{verbatim}

\section{Execution and Compilation}

\includegraphics[scale=0.35]{./Screenshot from 2022-11-20 19-52-50.png} 
\includegraphics[scale=0.35]{./Screenshot from 2022-11-20 19-53-12.png} 
\section{Profiling}
\includegraphics[scale=0.35]{Execution2.png} 
\includegraphics[scale=0.35]{./Screenshot from 2022-11-20 19-53-51.png} 
\section{Debugging}
BUG : Even after Buying multiple products the final receipt and
      the price are only of first product.
      
Error: The data.close() is written outside the for loop thus the           data is not resetting again thus executing once only.
\includegraphics[scale=0.35]{./Screenshot from 2022-11-20 22-27-53.png}    
\includegraphics[scale=0.35]{./Screenshot from 2022-11-20 22-27-59.png} 

    
 \end{document}